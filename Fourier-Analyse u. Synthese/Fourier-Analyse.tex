%Dokumentenklasse festlegen
\documentclass[paper=a4,titlepage]{scrartcl}

%Deutsche Spracheinstellungen
\usepackage[ngerman]{babel, varioref}
\usepackage[T1]{fontenc}
\usepackage[latin1]{inputenc}

%Mathematik Pakete
\usepackage{amsfonts}
\usepackage{amssymb}
\usepackage{amsmath}
\usepackage{amscd}
\usepackage{amstext}

%Programme
\usepackage{listings}
\lstset{literate=%
 {�}{{\"O}}1 
 {�}{{\"A}}1 
 {�}{{\"U}}1 
 {�}{{\ss}}2 
 {�}{{\"u}}1 
 {�}{{\"a}}1 
 {�}{{\"o}}1
 }

%Zitate
\usepackage{cite}

%Lange Tabellen sehen gut aus
\usepackage{longtable}
\usepackage{booktabs}

\usepackage{prettyref}

%Formatiert URLs
\usepackage{url}
\urlstyle{tt}

%Kontrollstrukturen laden
\usepackage{ifthen}

%Bilder
\usepackage{graphicx}

%\usepackage[usenames,dvipsnames]{pstricks}
\usepackage{epstopdf}
%\usepackage{pst-grad} % For gradients
%\usepackage{pst-plot} % For axes

%F�r die E-Mail-Adressen auf der Titelseite
\usepackage{footnpag}

%Links erlauben
\usepackage{hyperref}

%Einheitenklammern
\usepackage[ 
    separate-uncertainty  = true, 
    mode = text, 
    output-decimal-marker={.}, 
    repeatunits           = false, 
    range-phrase          = {\,bis\,}, 
]{siunitx}

%Kleiner Fix f�r psychologische Probleme mit dem Totenkreuz des 'Thanks'-Befehls
%Quelle: http://www.apfeltalk.de/forum/latex-thanks-titlepage-t94078.html
\makeatletter
\renewcommand{\@fnsymbol}[1]{\ensuremath{%
   \ifcase#1\or *\or **\or {**}*\or
   \mathsection\or \mathparagraph\or \|\or \star\or
   \star\star\or {\star\star}\star \else\@ctrerr\fi}}
\makeatother

%H�bsche Vektorpfeile mit vv{}
\usepackage{esvect}

%Einige Abk�rzungen f�r die Mathematik
\newcommand{\R}{\mathbb{R}}
\newcommand{\Z}{\mathbb{Z}}
\newcommand{\Q}{\mathbb{Q}}
\newcommand{\cbrt}[1]{\sqrt[3]{#1}}
\newcommand{\GNU}{{\sc{GNU}}Plot} % Sollte nicht mehr verwendet werden, besser \kap (Kapit�lchen)
\newcommand{\kap}[1]{{\sc{#1}}}

%\pic{filename.png}{desc}{label}{scale}
\newcommand{\pic}[4]{
    \begin{figure}[hptb]
      \begin{center}
	\includegraphics[width=#4\textwidth]{#1}
      \end{center}
      \caption{#2}
      \label{#3}
    \end{figure}
}


%Neue Formel-Nummerierung
\renewcommand{\theequation}{\arabic{section}.\arabic{subsection}.\arabic{equation}}

%Chem. Formel
\newcommand{\chem}[3]{${}^{#2}_{#3}\text{#1}$}

%Namen der Autoren
\newcommand{\flo}{Florian Reiche\thanks{\href{florian.reiche@tu-dortmund.de}{florian.reiche@tu-dortmund.de}}}
\newcommand{\jus}{Justin Grewe\thanks{\href{justin.grewe@tu-dortmund.de}{justin.grewe@tu-dortmund.de}}}

%Titelseite und Inhaltsverzeichnis generieren
\newcommand{\myTitlepage}[3]
{
\title{Protokoll zum Versuch Nr.{#1}: {#2}}
\author%{\flo \and \jus}
{Justin Grewe\\
{justin.grewe@tu-dortmund.de}
\and Florian Reiche\\
{florian.reiche@tu-dortmund.de}}
\date{#3}
\maketitle
\tableofcontents
\newpage
}

\begin{document}

\myTitlepage{351}{Fourier-Analyse und Synthese}{09.11.2012}

\section{Einleitung}
Periodische Funktionen lassen sich nach dem Fourierschen Theorem immer in Sinus- und Cosinus-Funktionen zerlegen. In diesem Versuch sollen periodische, elektrische Schwingungen in ihre Fourier-Komponenten zerlegt werden und die Ergebnisse mit dem theoretischen Werten mit Hilfe des Fourierschen Theorems verglichen werden. Umgekehrt sollen durch �berlagerung von Sinus-Funktionen bestimmte Schwingungen, wie die S�gezahn-Schwingung, erzeugt werden.
\section{Theorie}
Um die Fourierkoeffizienten zu untersuchen, wird die zeitabh�ngigen Funktionen zun�chst in den Frequenzraum transformiert. Dazu wird die Fouriertransformation benutzt:
\begin{equation}
g(v)=\int_{-\infty}^{\infty}f(t)e^{-ivt}dv\label{fourier1}
\end{equation}
Im Frequenzraum lassen sich die einzelnen Fourierkomponenten untersuchen. Die Fouriertransformation hat den Nachteil, dass man in der Praxis nicht �ber einen unendlichen Zeitraum integrieren kann, was dazu f�hrt, dass keinen diskreten Maxima entstehen die von Nebenmaxima begleitet werden. Damit die gemessenen Fourierkoeffizenten mit den theoretischen Werten verglichen werden k�nnen, m�ssen die Theoriewerte bestimmen werden. Diese lassen sich mit Hilfe des Fourierschen Theorems bestimmen:
\begin{equation}
S_f(t)=\frac{a_0}{2}+\sum^{\infty}_{n=}(a_n cos (n\dfrac{2\pi}{T}t) + b_n sin (n \dfrac{2\pi}{T}t))
\end{equation}
F�r die Koeffizienten $a_n$ und $b_n$ gilt:
\begin{equation}
a_n = \frac{2}{T}\int^{T}_{0}f(t)cos(n\dfrac{2\pi}{T}t)dt
\end{equation}
\begin{equation}
b_n = \frac{2}{T}\int^{T}_{0}f(t)sin(n\dfrac{2\pi}{T}t)dt
\end{equation}
In diesem Versuch werden die drei Signale Rechteckspannung, S�gezahn- und Dreiecksspannung untersucht. Durch das Fouriersche Theorem lassen sich die Koeffizienten errechnen.
F�r die Rechteckspannung gilt:
\begin{equation}
S_f(t) = A (\dfrac{sin(t)}{1}+\dfrac{sin(3t)}{3}+\dfrac{sin(5t)}{5}+...)
\end{equation}
A, B und C sind Konstanten und sind von der Amplitude der Schwingung abh�ngig. Der erste Term steht f�r die Grundschwingung und die folgenden f�r die jeweilige Oberschwingung. Bei der Rechteckspannung treten weitere Maxima bei der dreifachen (5, 7, 9,...) Frequenz auf, die Amplitude der Frequenzen soll um den Faktor drei (5, 7, 9...) geringer sein. F�r geradzahlige Vielfache der Grundfrequenz sollten keine Maxima auftreten.
F�r die S�gezahnspannung gilt:
\begin{equation}
S_f(t) = B (\dfrac{sin(t)}{1}+\dfrac{sin(t)}{2}+\dfrac{sin(3t)}{3}+...)
\end{equation}
Bei der S�gezahnspannung treten weitere Maxima bei der doppelten (3, 4, 5,...) Frequenz auf, die Amplitude der Frequenzen soll um den Faktor zwei (3, 4, 5...) geringer sein.

F�r die Dreiecksspannung gilt:
\begin{equation}
S_f(t) = C (\dfrac{cos(t)}{1^2}+\dfrac{cos(3t)}{3^2}+\dfrac{cos(5t)}{5^2}+...)
\end{equation}
Bei Dreiecksspannung treten weitere Maxima bei der dreifachen (5, 7, 9,...) Frequenz auf, die Amplitude der Frequenzen soll um den Faktor $3^2 (5^2, 7^2, 9^2,...)$ geringer sein. F�r geradzahlige Vielfache der Grundfrequenz sollten keine Maxima auftreten.\\
Mit den Koeffizenten werden Vergleiche zwischen Theorie und Praxis durchf�hrt. 
\section{Durchf�hrung}
\subsection{Versuch zur Fourier-Analyse}
Im ersten Teil des Versuchs, wird die Fourier-Transformation untersucht. Mittels einer Schaltung, bestehend aus regelbarem Funktionsgenerator und Oszilloskop(Abbildung 1 im Anhang),l�sst sich eine solche Betrachtung durchf�hren. Mit dem Funktionsgenerator wird eine Signalspannung erzeugt, die mit dem Oszilloskop dargestellt werden kann. Auf dem Oszilloskop wird au�erdem die Fouriertransformation angezeigt. Mit Hilfe des Cursors des Oszilloskops lassen sich nun die ben�tigten Maxima der Amplituden der Oberwellen ausmessen.
\subsection{Versuch zur Fourier-Synthese}
Im zweiten Teil des Versuchs, wird eine Fourier-Synthese ausgef�hrt. Es kommt eine Schaltung aus Oberwellengenerator und Oszilloskop zum Einsatz, au�erdem wird ein AC- Millivoltmeter eingesetzt(Abbildung im Anhang). Zuerst muss der Oberwellengenerator justiert werden. Zur Kontrolle der Phasenbeziehung der Fourier-Komponenten wird die erste Oberwelle auf den einen und die n-te Oberwelle auf den zweiten Eingang des Oszillographen gegeben, schaltet man nun auf den Zwei-Kanal-Betrieb schaltet, erscheint eine geschlossene Kurve auf dem Bildschirm, die sich Lissajous-Figur nennt. Durch Phasenverschiebung l�sst sich die geschlossene Kurve zu einer Kurve mit zwei Endpunkten entarten. Bei ungeradem n hat sich so die Phase 0 oder $\pi$ eingestellt. Bei geraden n und Sinus-Funktionen ist das Verh�ltnis umgekehrt und es stellt sich bei einer Phase von 0 oder $\pi$ die Lissajous-Figur ein. Sind alle Oberwellen in Phase, m�ssen mit Hilfe des AC-Mullivoltmeters die Amplituden gem�� der Fourier-Koeffizienten eingestellt werden um die Justierung des Oberwellengenerators abzuschlie�en. Nun werden Oberwellengenerator und Oszillograph wieder verbunden. Durch St�ck weises Zuschalten einzelner Oberwellen l�sst sich eine �berlagerungsfigur beobachten die sich der tats�chlichen Kurve ann�hert. Ist keine weitere Ann�herung m�glich wird ein Screenshot gemacht und die n�chste Signalspannung eingestellt.

\newpage
\section{Anhang}
\subsection{Abbildung zum Versuch zur Fourier-Analyse}
\pic{Schaltung1.eps}{Schematische Darstellung des Versuchsaufbaus zur Fourier-Analyse.}{pic:aufbau1}{1}
\subsection{Abbildung zum Versuch zur Fourier-Synthese}
\pic{Schaltung2.eps}{Schematische Darstellung des Versuchsaufbaus zur Fourier-Synthese.}{pic:aufbau2}{1}
\end{document}
