%Dokumentenklasse festlegen
\documentclass[paper=a4,titlepage]{scrartcl}

%Deutsche Spracheinstellungen
\usepackage[ngerman]{babel, varioref}
\usepackage[T1]{fontenc}
\usepackage[latin1]{inputenc}

%Mathematik Pakete
\usepackage{amsfonts}
\usepackage{amssymb}
\usepackage{amsmath}
\usepackage{amscd}
\usepackage{amstext}

%Programme
\usepackage{listings}
\lstset{literate=%
 {�}{{\"O}}1 
 {�}{{\"A}}1 
 {�}{{\"U}}1 
 {�}{{\ss}}2 
 {�}{{\"u}}1 
 {�}{{\"a}}1 
 {�}{{\"o}}1
 }

%Zitate
\usepackage{cite}

%Lange Tabellen sehen gut aus
\usepackage{longtable}
\usepackage{booktabs}

\usepackage{prettyref}

%Formatiert URLs
\usepackage{url}
\urlstyle{tt}

%Kontrollstrukturen laden
\usepackage{ifthen}

%Bilder
\usepackage{graphicx}

%\usepackage[usenames,dvipsnames]{pstricks}
\usepackage{epstopdf}
%\usepackage{pst-grad} % For gradients
%\usepackage{pst-plot} % For axes

%F�r die E-Mail-Adressen auf der Titelseite
\usepackage{footnpag}

%Links erlauben
\usepackage{hyperref}

%Einheitenklammern
\usepackage[ 
    separate-uncertainty  = true, 
    mode = text, 
    output-decimal-marker={.}, 
    repeatunits           = false, 
    range-phrase          = {\,bis\,}, 
]{siunitx}

%Kleiner Fix f�r psychologische Probleme mit dem Totenkreuz des 'Thanks'-Befehls
%Quelle: http://www.apfeltalk.de/forum/latex-thanks-titlepage-t94078.html
\makeatletter
\renewcommand{\@fnsymbol}[1]{\ensuremath{%
   \ifcase#1\or *\or **\or {**}*\or
   \mathsection\or \mathparagraph\or \|\or \star\or
   \star\star\or {\star\star}\star \else\@ctrerr\fi}}
\makeatother

%H�bsche Vektorpfeile mit vv{}
\usepackage{esvect}

%Einige Abk�rzungen f�r die Mathematik
\newcommand{\R}{\mathbb{R}}
\newcommand{\Z}{\mathbb{Z}}
\newcommand{\Q}{\mathbb{Q}}
\newcommand{\cbrt}[1]{\sqrt[3]{#1}}
\newcommand{\GNU}{{\sc{GNU}}Plot} % Sollte nicht mehr verwendet werden, besser \kap (Kapit�lchen)
\newcommand{\kap}[1]{{\sc{#1}}}

%\pic{filename.png}{desc}{label}{scale}
\newcommand{\pic}[4]{
    \begin{figure}[hptb]
      \begin{center}
	\includegraphics[width=#4\textwidth]{#1}
      \end{center}
      \caption{#2}
      \label{#3}
    \end{figure}
}


%Neue Formel-Nummerierung
\renewcommand{\theequation}{\arabic{section}.\arabic{subsection}.\arabic{equation}}

%Chem. Formel
\newcommand{\chem}[3]{${}^{#2}_{#3}\text{#1}$}
